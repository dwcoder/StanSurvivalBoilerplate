\documentclass[12pt]{article}

\usepackage[utf8]{inputenc}
\usepackage[T1]{fontenc}
\usepackage[english]{babel}

\usepackage{color}
\usepackage[table,usenames,dvipsnames]{xcolor}
\usepackage{graphicx}

\usepackage{amsmath,amssymb,amsthm,mleftright}
\usepackage{mathtools} % For cases environment
%\usepackage[a4paper]{geometry}
\usepackage[round]{natbib}
\usepackage{float}
\usepackage{nicefrac}
\usepackage{placeins} % This gives the \FloatBarrier command
\usepackage{setspace}
\usepackage{microtype}
\usepackage{url}
\usepackage{multicol}
\usepackage{minted}
\usepackage{titlesec} %to singlespace the headings

\usepackage{subcaption}
\usepackage[minmargin=0.1\textwidth,labelfont=bf, font=footnotesize, labelsep=period]{caption}
%\captionsetup[subfigure]{minmargin=0.1\textwidth}

% hyperref
% Got these values from Stan manual
\usepackage[%backref=page, 
            pdfpagelabels,
            plainpages=false,
            hypertexnames=true,
            colorlinks=true,
            linkcolor=linkcolor,
            anchorcolor=linkcolor,
            citecolor=linkcolor,
            filecolor=linkcolor,
            menucolor=linkcolor,
            runcolor=linkcolor,
            urlcolor=linkcolor,
            pdfborder={0 0 0} , 
            hidelinks]{hyperref} 


%to singlespace the headings

\titleformat{\chapter}[display]
{\singlespacing\normalfont\Huge\bfseries\raggedright}{\chaptername\ \thechapter}{1em}{}
\titleformat{\section}
{\singlespacing\normalfont\Large\bfseries}{\thesection}{1em}{}
\titleformat{\subsection}
{\singlespacing\normalfont\large\bfseries}{\thesubsection}{1em}{}
\titleformat{\subsubsection}
{\singlespacing\normalfont\normalsize\bfseries}{\thesubsubsection}{1em}{}

% % File that fixes and activates the line numbers
% \input{fixlinenumbers.tex}

% % showkeys
% \usepackage[notcite,notref]{showkeys}
% \renewcommand\showkeyslabelformat[1]{%
% {\normalfont\scriptsize\ttfamily#1}}

% Also from Stan manual
\definecolor{linkcolor}{RGB}{0,0,128}

% Setup the fancyvrb (which is called by minted)
\fvset{frame=single}
% Fix to make the code single spaced
% Took it from the fancyvrb manual and added my own fix
\makeatletter
\def\FV@SetupFont{%
 \FV@BaseLineStretch
 \ifx\@currsize\small\normalsize\else\small\fi\@currsize
 \linespread{1} % this makes the verbatim fonts single spaced
 \FV@FontSize
 \FV@FontFamily}
\makeatother


%\renewcommand{\familydefault}{\sfdefault} % change the font to sans serif

% make the quotes become singlespacing
\expandafter\def\expandafter\quote\expandafter{\quote\singlespacing}
\expandafter\def\expandafter\quotation\expandafter{\quotation\singlespacing}

\interfootnotelinepenalty=10000 % to ensure footnotes stay on the same page

% \allowdisplaybreaks[1] %{amsmath} -- allows formulas to span multiple pages


%DW's commands
\renewcommand{\vec}{\boldsymbol}
\DeclareMathOperator{\E}{\mathrm{E}}
\DeclareMathOperator{\Var}{\mathrm{Var}}


\title{
Survival models in Stan
}
\author{D.W. Bester}
\date{}

\begin{document}


\maketitle

\section{Model descriptions}

\subsection*{Model 1}\label{subsec:model-1}

This is a survival model, where events happen according to the the
hazard rate:
\begin{align*}
h(t) 
&= B e^{\theta t} \\
H(t) 
&= \int_0^t{h(x)dx} \\
&= \frac{B}{\theta} \left( e^{\theta t}-1  \right)
\end{align*}
The events are simulated using $H^{-1}(t)$, and then subjected to
random (uninformative) censoring.


\subsection*{Model 2}\label{subsec:model-2}

This is the same as model 1, but we have multiple gompertz parameters.
We try to use a \texttt{group} variable to assign the right gompertz
parameters within Stan.
Thus, this model fits the hazard rate
\begin{equation}
h(t) = B_j e^{\theta_j t}
\end{equation}
where $j$ is the group number of the subject in question. It is a
covariate in our model.

\subsection*{Model 3}\label{subsec:model-3}

This is the same as model 2, but we add a ``linear predictor'':
\begin{equation}
\eta =  \beta x
\end{equation}
where $x$ is a covariate.
The hazard rate then becomes
\begin{align*}
h(t)
&= B_j e^{\theta_j t} e^{\eta} \\
&= B_j e^{\theta_j t} e^{ \beta x }
\end{align*}
where $j$ is the group number of the subject in question. This is
similar to the cox proportional hazards model. Notice that the linear
predictor does not contain an intercept term. This is since the gompertz
parameter $B_j$ already acts as an intercept for that group. The
$\beta$ parameter measures the influence of of covariate $x$
measured against the baseline. That is, we can rewrite the hazard rate
as:
\begin{equation*}
h(t) = e^{ \theta_j t  + \beta x + log\left(B_j\right) }
\end{equation*}
Thus, $log\left(B_j\right)$ is the intercept for group $j$,
$\theta_j$ measures the influence of time on the hazard rate for group
$j$, and $\beta$ measures the influence of covariate $x$, assuming
the influence is global over all groups.

\section{Likelihood, estimation, and choice of prior
parameters}

All of our models assume that survival times come from a process with
hazard rate $h(t)$, and we can use a Poisson process to show the
likelihood contribution of each observation. Let $T$ be the event-time
of a subject in our model, then
\begin{equation*}
T > t \quad \longleftrightarrow \quad N(t) < 1
\end{equation*}
where $N(t)$ is the number of events at time $t$, which will always
be 0 or 1 in our model, then we have
\begin{align*}
N(t) \sim \text{Poisson}\left( \int_0^t{h(x)dx} \right)
\text{.}
\end{align*}
We don't observe $T_i$ directly, since our observations are subject to
censoring. We observe $V_i = \text{min}(T_i, C_i)$ where $C_i$ is a
censoring time, assumed to be random and independent of $T$
(non-informative censoring). Our observations are the pairs
$(v_i, \delta_i)$ where $\delta_i$ is an indicator taking 1 if
$T_i \leq C_i$ and 0 otherwise. Thus, $\delta_i$ is an event
indicator, and it will be equal to the number of events at time $t$.

The Poisson process (and some algebra) provides the the likelihood
contribution of the observation pair $(v_i, \delta_i)$:
\begin{align*}
\mathcal{L}\left(\theta |(v_i,\delta_i) \right)
&= S(v_i)^{\delta_i} f(v_i)^{(1-\delta_i)} \\
&= h(v_i)^{\delta_i} \exp{\left( - \int_0^{v_i}{h(x)dx} \right)} \\
&= h(v_i)^{\delta_i} \exp{\left( - H(v_i) \right)}
\end{align*}
where $\theta$ is the set of all parameters. This is the likelihood we
need to add to MCMC software if we want to fit survival
models.\footnote{MCMC software like Stan and JAGS do provide Poisson
  densities, but we can only use them if the hazard rate is constant.
  For more complicated hazard rate functions, we have to resort to more
  tedious measures. Since the likelihood contains both $h(t)$ and $H(t)$, we need to add these time-varying functions to the sampler.} The point of this repository is to show how this can done using Stan's \texttt{functions} block.

Stan was able to recover the chosen parameters for all of the above
models. For model 3, the more complicated of the three models in terms
of number of parameters, convergence was very sensitive to the choice of
priors assumed on the Gompertz parameters, $\vec{B}$ and
$\vec{\theta}$.

\emph{Add discussion here about the nuance of the prior parameters, how
it effected the convergence of model 2 vs model 3, and how to avoid this
problem in general by thinking carefully about prior parameters.}


\end{document}
